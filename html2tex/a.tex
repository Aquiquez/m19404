\documentclass[fontsize=12pt, % Document font size
                             paper=a4, % Document paper type
                             oneside, % Shifts odd pages to the left for easier reading when printed, can be changed to oneside
                             captions=tableheading,
                             index=totoc,
                             hyperref]{labbook}
 
\usepackage[bottom=10em]{geometry} % Reduces the whitespace at the bottom of the page so more text can fit
\usepackage[english]{babel} % English language
\usepackage{lipsum} % Used for inserting dummy 'Lorem ipsum' text into the template
\usepackage[T1]{fontenc} % Use 8-bit encoding that has 256 glyphs
\usepackage{emerald}
\usepackage{booktabs,array} % Packages for tables
\usepackage{amsmath} % For typesetting math
\newcommand{\fullpage}[1]{
\begin{frame}
 #1
\end{frame}
}
\usepackage{movie15}  
\usepackage{etoolbox}
\usepackage[norule]{footmisc} % Removes the horizontal rule from footnotes
\usepackage{lastpage} % Counts the number of pages of the document
\usepackage{color}
\usepackage{listings}
\definecolor{Code}{rgb}{0,0,0}
\definecolor{Decorators}{rgb}{0.5,0.5,0.5}
\definecolor{Numbers}{rgb}{0.5,0,0}
\definecolor{MatchingBrackets}{rgb}{0.25,0.5,0.5}
\definecolor{Keywords}{rgb}{0,0,1}
\definecolor{self}{rgb}{0,0,0}
\definecolor{Strings}{rgb}{0,0.63,0}
\definecolor{Comments}{rgb}{0,0.63,1}
\definecolor{Backquotes}{rgb}{0,0,0}
\definecolor{Classname}{rgb}{0,0,0}
\definecolor{FunctionName}{rgb}{0,0,0}
\definecolor{Operators}{rgb}{0,0,0}
\definecolor{Background}{rgb}{0.98,0.98,0.98}
\renewcommand{\familydefault}{\sfdefault}
\usepackage[dvipsnames]{xcolor}  
\definecolor{titleblue}{rgb}{0.16,0.24,0.64} 
\definecolor{linkcolor}{rgb}{0,0,0.42} 
\addtokomafont{title}{\ECFAugie\color{Mahogany}} % Titles in custom blue color
\addtokomafont{chapter}{\ECFAugie\color{LimeGreen}} % Lab dates in olive green
\addtokomafont{section}{\ECFAugie\color{Cyan}} % Sections in sepia
\addtokomafont{subsection}{\ECFAugie\color{LimeGreen}} % Sections in sepia
\addtokomafont{subsubsection}{\ECFAugie\color{Orange}} % Sections in sepia
\addtokomafont{pagehead}{\ECFAugie\color{Red}} % Header text in gray and sans serif
\addtokomafont{caption}{\ECFAugie\footnotesize\itshape} % Small italic font size for captions
\addtokomafont{captionlabel}{\ECFAugie\upshape\bfseries} % Bold for caption labels
\addtokomafont{descriptionlabel}{\ECFAugie}
\setcapwidth[r]{10cm} % Right align caption text
\setkomafont{footnote}{\sffamily} % Footnotes in sans serif
\deffootnote[4cm]{4cm}{1em}{\textsuperscript{\thefootnotemark}} % Indent footnotes to line up with text
\DeclareFixedFont{\textcap}{T1}{fjd}{bx}{n}{1.5cm} % Font for main title: Helvetica 1.5 cm
\DeclareFixedFont{\textaut}{T1}{fjd}{bx}{n}{0.8cm} % Font for author name: Helvetica 0.8 cm
\usepackage[nouppercase,headsepline]{scrpage2} % Provides headers and footers configuration
\pagestyle{scrheadings} % Print the headers and footers on all pages
\clearscrheadfoot % Clean old definitions if they exist
\automark[chapter]{chapter}
\ohead{\headmark} % Prints outer header
\setlength{\headheight}{25pt} % Makes the header take up a bit of extra space for aesthetics
\setheadsepline{.4pt} % Creates a thin rule under the header
\addtokomafont{headsepline}{\ECFAugie\color{lightgray}} % Colors the rule under the header light gray
\ofoot[\normalfont\normalcolor{\thepage\ |\  \pageref{LastPage}}]{\normalfont\normalcolor{\thepage\ |\  \pageref{LastPage}}}
\makeatletter
\patchcmd{\addchap}{\if@openright\cleardoublepage\else\clearpage\fi}{\par}{}{}
\makeatother
\renewcommand*{\chapterpagestyle}{scrheadings}
\usepackage{chngcntr}
\counterwithout{figure}{labday}
\counterwithout{equation}{labday}
\definecolor{dkgreen}{rgb}{0,0.6,0}
\definecolor{gray}{rgb}{0.5,0.5,0.5}
\definecolor{mauve}{rgb}{0.58,0,0.82}
\lstset{ %
  language=C++,                % the language of the code
  basicstyle=\footnotesize\ttfamily,           % the size of the fonts that are used for the code
  numbers=left,                   % where to put the line-numbers
  numberstyle=\tiny\color{gray},  % the style that is used for the line-numbers
  stepnumber=1,                   % the step between two line-numbers. If it's 1, each line 
  numbersep=5pt,                  % how far the line-numbers are from the code
  backgroundcolor=\color{black},      % choose the background color. You must add \usepackage{color}
  showspaces=false,               % show spaces adding particular underscores
  showstringspaces=false,         % underline spaces within strings
  showtabs=false,                 % show tabs within strings adding particular underscores
  frame=single,                   % adds a frame around the code
  rulecolor=\color{black},        % if not set, the frame-color may be changed on line-breaks within not-black text (e.g. comments (green here))
  tabsize=2,                      % sets default tabsize to 2 spaces
  captionpos=b,                   % sets the caption-position to bottom
  breaklines=true,                % sets automatic line breaking
  breakatwhitespace=false,        % sets if automatic breaks should only happen at whitespace
  title=\lstname,                   % show the filename of files included with \lstinputlisting;
  keywordstyle=\color{dkgreen},          % keyword style
  commentstyle=\color{dkgreen},       % comment style
  stringstyle=\color{mauve},         % string literal style
  escapeinside={\%*}{*)},            % if you want to add LaTeX within your code
  morekeywords={Rect,...},              % if you want to add more keywords to the set
  deletekeywords={...}              % if you want to delete keywords from the given language
}
\usepackage[
    pdfauthor={pi19404}, % Your name for the author field in the PDF
    urlcolor=Goldenrod, % Color of URLs
    citecolor=Goldenrod, % Color of citations
    linkcolor=Goldenrod, % Color of links to other pages/figures
    pdfpagelabels=true,
    plainpages=false,
    colorlinks=true, % Turn off all coloring by changing this to false
    bookmarks=true,
    pdfview=FitB]{hyperref}
\usepackage{caption}
\usepackage{subcaption}
\usepackage{movie15}
\usepackage{filecontents}
\usepackage[natbib=true,citestyle=authoryear,bibstyle=numeric]{biblatex} 
\usepackage{graphicx}
\DeclareGraphicsExtensions{.jpg,.jpeg,.pdf,.png,.mps,.gif}
\usepackage{animate}
\usepackage{eso-pic}
\newcommand\BackgroundPic{
\put(0,0){
\parbox[b][\paperheight]{\paperwidth}{
\vfill
\centering
\includegraphics[width=\paperwidth,height=\paperheight]{../back.jpg}
\vfill
}}}
\renewcommand*{\compcitedelim}{\addsemicolon\space}
\usepackage{minted}
\ECFAugie\color{lightgray}
\bibliography{test} 
\begin{document}
\AddToShipoutPicture{\BackgroundPic}
\ECFAugie\color{lightgray}
\title{\textcap{Motion Planning : Bug Algorithm \\[2cm]  
%\textaut{Beginning 30-05-2012}
}}
\author{
    \textaut{Pi19404}\\ \\ % Your name
    %Master of Science % Your degree
}
%\ECFJD{\today} % No date by default, add \today if you wish to include the publication date

\maketitle % Title page
\printindex
\tableofcontents % Table of contents
\newpage % Start lab look on a new page
%
%\begin{addmargin}[0cm]{0cm} % Makes the text width much shorter for a compact look
%
\pagestyle{scrheadings} % Begin using headers
%
\ECFAugie\color{White}
\labday{Motion Planning : Bug Algorithm}

%\ECFTeenSpirit
%\section{Augie} {\ECFJD\lipsum[4]}
%\section{Auriocus Kalligraphicus} {\Fontauri\lipsum[4]}
%\section{BrushScriptX-Italic} {\bsifamily\lipsum[4]}

\section{Introduction}
In the article we will look at implementation of bug 2 algorithm for motion planning
\section{Bug Algorithms}
The aim of path planning algorithm is to complete a collision free path from initial to goal position.
Bug algorithms are simplest type of path planning algorithms.
\\\\
In Bug algorithms no global model of the world is assumed ,the location and shapes
of the obstacles are unknown.Only information
acquired through sensing is known.The Bug algorithms assume local knowledge of environment and a global goal.
\\\\
The environment is a Two-dimensional scene filled with unknown obstacles.
The perimeter of any obstacle is finite, and that the number of obstacles is finite
and can be of arbitrary shape.
\\\\
We consider a point robot that moves among arbitrary
obstacles in a planar environment. We assume that the workspace is bounded. 
\\\\
The robot is employed with ideal localization module.The robot can always keep track of position of robot.
Thus we can always measure the distance $d(s,g)$ between the source and goal.
\\\\
The robot is employed with tactile touch or a range sensor sensor.The touch sensor can detect
contact with the object .
\\\\

Sample workspace configurations are shown in \ref{fig:1}.The red dot is robot center
and green dot is the goal.
\begin{figure}[!htbp]
\begin{subfigure}[b]{0.5\textwidth}
                 \centering
                 \includegraphics[width=\textwidth]{im1.png}
                 \caption{Workspace}
\end{subfigure}                
\begin{subfigure}[b]{0.5\textwidth}
                 \centering
                 \includegraphics[width=\textwidth]{im2.png}
                 \caption{Workspace 2}
  		 
\end{subfigure}  
\caption{Workspace}
\label{fig:1}
\end{figure}
\subsection{Bug algorithm}
The robot is equipped with a range sensor.Range sensor with a small radius 
can be considered to be equivalent to a tactile sensor.
\\\\
At any given point of time robot is assumed to be in one
of following low level states \texttt{Motion,Boundary Detection,Navigate Boundary,Done}
\\\\
\section{Motion}
In this mode the robot moves along a specified heading direction which is specified
in terms of angle wrt global co-ordinate system.
\\\\
At initialization we have information only about the source and the goal position.
The shortest distance between two points is a straight line.
The heading direction is along this line from source towards the goal.
\\\\
The robot begins  to move along the straight line towards the goal.
\\\\
To specify displacement to robot to move along the x and y directions,
the unit vectors along the heading direction are computed.Let $\delta x,\delta y$
denote the unit vectors
\\\\
Then robot take a small displacement in the heading direction 
by taking a small steps proportional to components of unit vector along
the x and y directions.
\begin{eqnarray}
  x_{t+1} = x_t + \delta{x} \\
  y_{t+1} = x_y + \delta{y} \\
  \delta{D}= (\delta{x},\delta{y})
\end{eqnarray}
Each time position is incremented the displacement is recorded
Let us call this displacement $\delta{D}$.
\begin{minted}{cpp}
//function that perform motion towards the goal operation
     void Motion()
    {
        if(m.value==m.Motion)
        {
	//compute components of unit vector along the heading direction        
        float dx=cos(heading*PI/180); 
        float dy=sin(heading*PI/180);
        //update the current position
        position.x=position.x+1*dx;
        position.y=position.y+1*dy;
        //store the change in postion
        prevdelta=FPoint2f(dx,dy);
        }
    }
\end{minted}

\section{Boundary Detection}
The robot is equipped with a range sensor.The sensor 
range is has a finite radius R.If a obstacle is encountered
the range sensor gives the direction and distance to the object.
Let is assume that direction resolution of sensor is 1 degree.
\\\\
Thus if a range sensor detects some object lying within
the radius R.The robot enters the boundary detection state.
\\\\
Since range is limited.The robot cannot estimate if
single or multiple obstacles are encountered.
\\\\
It now enters the navigate and starts moving along the heading direction.
\\\\
\begin{minted}{cpp}
void Boundary()
{
  if(m.value==m.Boundary)
  {
//change the heading direction by 90 deg
//to best scan line
for(int i=0;i<scan.size();i++)
{
  if(scan[i].index==best_index)
  {	
    if(scan[i].position.x!=-1 && scan[i].position.y!=-1)
    {
      heading=best_index+90;
      //change state to boundary navigation
      m.value=m.Nav;
      return;
    }
    else
    {
      break;
    }
  }

}
  } 
\end{minted}

\section{Navigate Boundary}
Now the robot has to navigate along the boundary of the obstacle
till some predefined criteria is satisfied.
\\\\
Thus robot must change it heading direction.The direction
along with shortest distance to the boundary recorded by the range sensor is determined.
\\\\
The robot changes its heading along line perpendicular to
direction along shortest distance to the boundary point.
\\\\
In this stage the robot still can sense obstacle with its range
and it determines the heading direction at each instant as direction
perpendicular to direction along shortest distance to obstacle.
It continues to move along the heading direction.
\\\\
\begin{minted}{cpp}
if(scan.size()==0)
{
    //if robot moves away from boundary,try to go to previous position
    //move along the edge
     position.x-=prevdelta.x;
     position.y-=prevdelta.y;
     heading=best_index;
}
else
{
     //move direction perpendicular to best scan line
     heading=best_index+90;
}
//computing components of unit vector along heading
float dx=cos(heading*PI/180);
float dy=sin(heading*PI/180);
//compute the new position of robot
prevdelta=FPoint2f(dx,dy);
position.x=position.x+5*dx;
position.y=position.y+5*dy;
\end{minted}

While moving along the heading direction a point is reached where no
obstacles are encountered.
This can happen if robot has reached a sharp edge or it has moved
away from the boundary while navigating the boundary due to inaccuracies
in determining the shortest distance along the boundary.
\\\\
The new position of robot is determine on the same lines
as that described in motion towards the goal section.
At each instant the current heading direction and previous displacement
information is stored.
\\\\
We take a step back using $\delta{D}$ and go back 
to the last position.And begin the move in a direction perpendicular to the current heading direction.
This is done by remaining in the same state and changing the heading direction.
\\\\
This will cause the robot either to move back to boundary or move along
the edge.
\\\\
In this state the robot will circumnavigate the boundary of the obstacle.
\\\\
To move towards the goal,at some point based on some predefined
criteria robot will leave the navigation state and enter
the motion towards the goal state.
\\\\
Different bug algorithms differ in the way they define transition from
boundary following to motion towards goal.
%\\\\
%A path is defined as sequence of hit and leave points and different algorithm
%differ in computation of leave points.
%\\\\
%A different approach, which guarantees reaching
%the target if possible, was presented by Lumelsky and
%Stepanov [7]. The robot, equipped only with contact
%sensors, goes straight to the target until hitting an
%obstacle. It then follows the object boundary. The
%algorithms define “leaving conditions” that determine
%where to leave the obstacle boundary and go directly
%towards the target again. The convergence is based
%on the fact that the distance from every leave point
%to the target is strictly smaller than the distance from
%the corresponding hit point.

\section{Simulation Environment}
A basic simulation environment is developed to test the algorithm.
The input to the simulation environment is a image file containing
the obstacles.Thus obstacles of any complexity can be included
for testing.
\\\\
The sensor also needs to be simulated.Thus given a location
if a point lying along the circle with center at robot present location
lies withing the obstacle needs to be determined.Also how far
inside the obstacle the point lies determines the distance of
the robot from the obstacle.
\\\\
Simulation environment plots the obstacles,robot and goal positions,
trajectory,heading direction,direction along which obstacles are detected.
Debug information is also displayed along with plots.
\\\\
OpenCV libraries are used for plotting .To determine if a point lies inside
a polygon the function provided by opencv libraries are used.
\\\\
There are two types of obstacles defined boundary and normal obstacles.
The boundary obstacles just is boundary of simulation environment.
\\\\
Thus for boundary obstacles the simulation env determine if the point
on the circle lies outside the boundary while of other obstacles
it checks if point lies inside the boundary.
\\\\
\subsection{Sensor}
The sensor simulation if performed by finding point that lie on the circle
of sensor range $\mathcal{R}$.The test is performed if the point lies
inside/outside a polygon obstacles or not.All the direction along
circle are evaluated .The current heading direction is indexed 0.
\\\\
\begin{minted}{cpp}
for( int i = -180; i <= 179; i=i+delta)
{

    intersection = cos(toRadians((heading+i)))*sensorRange;
    intersectionPointWithSensorCircleY = sin(toRadians((heading+i)))*sensorRange;

    //point on the circle
    intersectionPointWithSensorCircleX+=robot.position.x;
    intersectionPointWithSensorCircleY+=robot.position.y;
    Point2f ff=Point2f(intersectionPointWithSensorCircleX,-intersectionPointWithSensorCircleY);
    
    //cycle over all the obstacles
    for(int k=0;k<o.size();k++)
    {
       Obstacle o1=o[k];


       double d=pointPolygonTest( o1.contours[0], ff,true);
       bool flag,flag1;
       //check if point in inside the normal obstacle
       if(o1.type==0)
       {
          flag=d>=0;
          best=-999999;
       }
       //check if point in outside the boundary obstacle
       else
       {
          flag=d<0;
          best=999999;
       }
	//if obstacle detected
	if(flag)
	{

	//add point to stack
	points.push_back(ScanPoint(FPoint2f(ff.x,-ff.y),heading+i));
	if(o1.type==0)
	   flag1=d>=best;
	   else
	   flag1=d<best;
	   //find closest distance to obstacle
        if(flag1)
	{
	  best=d;
	  besti=heading+i;
        }
	}
       
\end{minted}

\subsection{Bug 2 Algorithm}
%The location at which boundary is hit is recorded by the robot
%as a hit point.The line joining the hit point and the goal
%is defined as the MLine.
%\\\\
Bug Algorithms are greedy algorithms hence not predictible.
\\\\
At the initialization the MLine is defined as the line joining the starting
location and the goal.
\\\\
Leave point is point at which robot transitions from boundary following
to motion towards the goal states.
\\\\
In the Bug 2 algorithm the leave point is a point on the MLine encountered
after encountering hit point during boundary following.
\\\\
Leave point is a point at we are able to leave the surface of obstace and have
heading towards the goal.
\\\\
The leave point is corresponding point on the bounday at the other side of obstacle that lies 
along the line joining the hit point and goal is called as leave point.
\\\\
\begin{minted}{cpp}
//check if point on MLine
if(mline.isMLine(position)==true)
{
//detected hit point
	if(hl.size()==0)
	{
	hl.push_back(position);
	}
//possible leave point	
	else
	{
//distance from hit point to goal	
	    float d1=dist(mline.hit,goal.position);
//distance from leave point to goal	    
	    float d2=dist(position,goal.position);
//if leave point farther,not leave point	    
	    if(d2>d1)
		return;
//enter motition towards goal state		
	m.value=m.Motion;
//initialize heading	
	computeHeading();
//computer unit vectors along heading	
	float dx=cos(heading*PI/180);
	float dy=sin(heading*PI/180);
//update the position	
	position.x=position.x+3*dx;
	position.y=position.y+3*dy;
	prevdelta=FPoint2f(dx,dy);
//clear the stack	
	hl.pop_back();
	}
}

\end{minted}

In some situations it may happen that Bug2 algorithm takes you farther away from the target,
or it may be stuck in a loop and robot never reaches the target
\\\\
to avoid such situations a constraint is placed that the leave point should be closer
to the goal than the hit point.
\\\\
Let the distance between source to goal be denoted by $\mathcal{D}$
Let the perimeter of the $i'th$ obstacle be denoted by $P_i$
The shortest distance Robot will travel using bug2 algorithm is $\mathcal{D}$.
\\\\
Another variation is to check if there are obstacles along the direction to the goal.
This no obstacles are detected then change the state to motion towards the goal.
The freeheading flag is set if there are no obstacles in the direction of the goal.
However since the range/tactile sensor have limited range ,a free heading does not
necessarily indicate a free heading towards the goal.
\\\\
Additional constraints such that potential leave point must be close to the goal
that the previous hit point and on heading angle are imposed to ensure 
boundary following is exited at proper position.
\begin{minted}{cpp}
             freeheading=false;
             ScanPoint pp;
             for(int k=0;k<scan.size();k++)
             {
                 ScanPoint p=scan[k];
                 float h1=p.index+heading;
                 if(h1<=-180)
                     h1=h1+360;
                 if(h1>180)
                     h1=h1-360;
                 h1=(int)h1%360;

                 if(h1==cur_heading && p.distance==-1)
                 {

                     freeheading=true;
                     pp=p;
                 }

             }

             if(freeheading==true)
             {
                 //cv::waitKey(0);
                 float distance1=FPoint::dist(hl[0],goal.position);
                 float distance2=FPoint::dist(goal.position,pp.position);
                 if(distance1>distance2 && heading<cur_heading-30)
                 {
                 m.value=m.Motion;
                 computeHeading();
                 float dx=cos(heading*PI/180);
                 float dy=sin(heading*PI/180);
                //update the position
                 position.x=position.x+1*dx;
                 position.y=position.y+1*dy;
                 prevdelta=ScanPoint(FPoint2f(dx,dy),-1,0);
                 //cerr << ":" << cur_heading <<":" << heading << endl;
                 hl.pop_back();
                 }

             }

\end{minted}

\subsection{Analysis}
When the obstale encounters the bounday it will move in clockwise and anti clockwise
direction till it encounters the Mline again.Thus maximum distance it will traverse
is $\mathcal{P}_i/2$
\\\\
It may happen that robot intersects the same obstacle multiple times $n_i$.
\begin{eqnarray}
 D + \sum_i \frac{n_i \mathcal{P}_i}{2}
\end{eqnarray}
\\\\

%\subsection{Bug 1 Algorithm}
\section{Simulation}
The output simulation environments can be found at \\\\
vurl{out.ogv}\\\\
vurl{out-5.ogv}\\\\
\section{Code}
The code for the testing and training utility can be found at
curl{https://github.com/pi19404/m19404/tree/master/robot/navigation}
\section{PDF}
The PDF version of the document can be found at
durl{http://www.scribd.com/doc/163669836/Robotics-Bug-Algorithm-Simulation}



%\end{addmargin}

%\printbibliography
%\nocite{2,3}
%\embedfile{/media/sda7/SEMX/hand/test1/GaussianMixtureModel/color_balance1.cpp}
% %----------------------------------------------------------------------------------------
% %	BIBLIOGRAPHY
% %----------------------------------------------------------------------------------------
% 
%\begin{thebibliography}{1}

% 


%\end{thebibliography}

%----------------------------------------------------------------------------------------



\addcontentsline{toc}{section}{References}
%\printbibliography 
\nocite{*}
\end{document}

% @misc{a et~al(2007)a,
%       title = "Dynamic Time Warping",
%        note = {at \url{http://web.science.mq.edu.au/~cassidy/comp449/html/ch11s02.html}},
%        year = {2007},
% }
% @INPROCEEDINGS{b et~al(2007)b,
%     author = {Ricardo, Gutierrez-Osuna},
%     title = {Introduction to Speech Processing},
%     booktitle = {CSE@TAMU},    
%      year = {2007},
% }
